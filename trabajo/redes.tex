\documentclass[1p]{elsarticle}

\usepackage{lineno,hyperref}
\modulolinenumbers[5]
\usepackage[utf8]{inputenc}
\usepackage[spanish]{babel}
\usepackage{amsmath}
\usepackage{graphicx}
\usepackage{amsfonts}
\usepackage{amssymb}
\newtheorem{thm}{Teorema}
\newtheorem{lem}[thm]{Lema}
\newdefinition{rmk}{Remark}
\newproof{pf}{Demostración}
\newproof{pot}{Demostración del Teorema \ref{thm2}}
%%\bibliographystyle{IEEEannot}

%% `Elsevier LaTeX' style
\bibliographystyle{elsarticle-num}
%%%%%%%%%%%%%%%%%%%%%%%
\usepackage{setspace}  
\begin{document}

\begin{frontmatter}

\title{A review of the paper: \textit{Inherent directionality explains the lack of feedback loops in
	empirical networks} }

%% Group authors per affiliation:
\author{Rubén Hurtado, Bartolomé Ortiz, Cristina Seva}
\address{Master en Física y Matemáticas\\ Universidad de Granada\\10/06/2018}

\begin{abstract}
This work is a brief revision and study about a paper called: \textbf{Inherent directionality explains the lack of feedback loops in empirical networks} written by \textit{Virginia Domínguez García, Simone Pigolotti and Miguel A. Muñoz}  . Our aim is to present its mains results, some technical highlights related to the mathematical and physical advances, and reproduce a minor result using our own methods. It will be analized its implications and future research too. Althought this work is merely a revision it could be useful as a source of code to reproduce some of the results.
\end{abstract}

\begin{keyword}
 \texttt{complex networks} \sep \texttt{mathematics}\sep \texttt{complexity} \sep \texttt{graph theory}

\end{keyword}

\end{frontmatter}

\linenumbers

\section{Introducción}
\spacing{1.2}
El objetivo de este trabajo es presentar el artículo seleccionado \cite{arti}, de manera que expondremos sus principales descubrimientos, así como algunas notas sobre los procedimientos que en él se presentan. Tras esto, vamos a realizar un pequeño intento de simulación para intentar reproducir algunos de los resultados que aparecen en el articulo.
Finalmente, para cerrar nuestro trabajo, hablaremos un poco sobre las conclusiones obtenidas de este y sobre posibles vías para avanzar.

\section{Review del artículo}


\section{Intento de reproducción}


\section{Conclusiones finales}

\section*{Referencias}

\bibliography{bibliograf}

\end{document}